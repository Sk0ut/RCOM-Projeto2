\documentclass[11pt,a4paper,reqno]{article}
\linespread{1.5}

\usepackage[active]{srcltx}
\usepackage{listings}
\usepackage{graphicx}
\usepackage{amsthm,amsfonts,amsmath,amssymb,indentfirst,mathrsfs,amscd}
\usepackage[mathscr]{eucal}
\usepackage[active]{srcltx} %inverse search
\usepackage{tensor}
\usepackage[utf8x]{inputenc}
%\usepackage[portuges]{babel}
\usepackage[T1]{fontenc}
\usepackage{enumitem}
\setlist{nolistsep}
\usepackage{comment} 
\usepackage{tikz}
\usepackage[numbers,square, comma, sort&compress]{natbib}
\usepackage[nottoc,numbib]{tocbibind}
%\numberwithin{figure}{section}
\numberwithin{equation}{section}
\usepackage{scalefnt}
\usepackage[top=2.5cm, bottom=2.5cm, left=2.5cm, right=2.5cm]{geometry}
%\usepackage{tweaklist}
%\renewcommand{\itemhook}{\setlength{\topsep}{0pt}%
%	\setlength{\itemsep}{0pt}}
%\renewcommand{\enumhook}{\setlength{\topsep}{0pt}%
%	\setlength{\itemsep}{0pt}}
%\usepackage[colorlinks]{hyperref}
\usepackage{MnSymbol}
%\usepackage[pdfpagelabels,pagebackref,hypertexnames=true,plainpages=false,naturalnames]{hyperref}
\usepackage[naturalnames]{hyperref}
\usepackage{enumitem}
\usepackage{titling}
\newcommand{\subtitle}[1]{%
	\posttitle{%
	\par\end{center}
	\begin{center}\large#1\end{center}
	\vskip0.5em}%
}
\newcommand{\HRule}{\rule{\linewidth}{0.5mm}}
\usepackage{caption}
\usepackage{etoolbox}% http://ctan.org/pkg/etoolbox
\usepackage{complexity}

\usepackage[official]{eurosym}

\def\Cpp{C\raisebox{0.5ex}{\tiny\textbf{++}}}

\makeatletter
\def\@makechapterhead#1{%
  %%%%\vspace*{50\p@}% %%% removed!
  {\parindent \z@ \raggedright \normalfont
    \ifnum \c@secnumdepth >\m@ne
        \huge\bfseries \@chapapp\space \thechapter
        \par\nobreak
        \vskip 20\p@
    \fi
    \interlinepenalty\@M
    \Huge \bfseries #1\par\nobreak
    \vskip 40\p@
  }}
\def\@makeschapterhead#1{%
  %%%%%\vspace*{50\p@}% %%% removed!
  {\parindent \z@ \raggedright
    \normalfont
    \interlinepenalty\@M
    \Huge \bfseries  #1\par\nobreak
    \vskip 40\p@
  }}
\makeatother

\usepackage[toc,page]{appendix}

%\addto\captionsportuges{%
%  \renewcommand\appendixname{Anexo}
%  \renewcommand\appendixpagename{Anexos}
%}

%\addto\captionsportuges{%
%  \renewcommand\abstractname{Sumário}
%}

\begin{document}



\begin{titlepage}
\begin{center}
 
\vspace*{3cm}

{\Large Redes de Computadores}\\[2cm]

% Title
{\Huge \bfseries Configuration of a network and development of download application \\[1cm]}

% Author
{\large Flávio Couto, João Gouveia e Pedro Afonso Castro}\\[2cm]

\includegraphics[width=10cm]{feup_logo.jpg}\\[2cm]


% Bottom of the page
{\large \today}

\end{center}
\end{titlepage}

%%%%%%%%%%%
% SUMARIO %
%%%%%%%%%%%
\begin{abstract}
This report aims to explain our approach regarding the second project proposed by our teachers in the subject “Redes de Computadores”. This project was split into two parts. The development of a download application, using TCP sockets to connect to an FTP server to download the file requested by the user, and into the configuration and studying of a computer network, consisting in configuring an Internet Protocol (IP) network, implementing virtual LANs (VLANs) in a switch, configuring a router and a commercial router, implementing Network Address Translation (NAT) and configuring the Domain Name System (DNS), establishing a TCP connection (using the application developed in the first part of the project).

Our group managed to comprehend and reach all the purposed goals with success, as we will show in this report.

\end{abstract}

\tableofcontents
\newpage

%%%%%%%%%%%%%%
% INTRODUCAO %
%%%%%%%%%%%%%%
\section{Introduction}

	The purpose of this project is twofold: develop an application capable of downloading files from an FTP server through TCP sockets and implement a computer network. To implement the computer network, several equipment was used, namely a switch, a commercial router and a computer with Linux. The appliction, just like the serial port protocol implemented on the previous project, works on any computer capable of interpreting POSIX calls.
	
	In what concerns the report, its main objective is not only to describe the approach we used in this project, regarding both parts of the project, but also to show what we learnt and concluded from the development of this project. Tests are also provided in order to demonstrate the result of our work.
	
	This report is composed of an introduction, where the aim of the project and of this report are explained, the FTP Client section, where the application is described, a Network configuration section, where the computer netword is described, a conclusion, where we explain what we learnt from this project, evaluate our performance and our overall opinion regarding the project, and the annex section, where we place the code and some images that help with the comprehension of each part.

\section{FTP Client}

As previously stated, one of this project’s parts was developing an FTP Client capable of connecting to an FTP server, logging in (either with the credentials provided by the user, or in anonymous mode), and downloading a file requested by the user. Said FTP Client was developed in the C programming language, and to develop it several documents regarding the File Transfer Protocol had to be studied, namely RFCs (Request for Comments) 959, describing the FTP Protocol specification, and 1738, regarding URL syntax. In order to communicate with the FTP server, we resort to socket programming, more specifically, to TCP sockets.

	\subsection{Architecture of the download application}

	The application is essentialy split into two main modules: the URL Parsing Module and the FTP Client module. The URL Parsing module validates the URL, determining if it is an authenticated or anonymous connection, and parses it onto several variables to be used by the FTP Client, which makes the connection, authentication, file downloading and disconnection with said data. In addition to that, both modules have some debug messages shown in the terminal if the symbolic constant DEBUG is defined. In order to make them appear, just define said constant in the ftpClient.h file.

	\subsubsection{The URL Parsing Module}

	As previously mentioned, the URL Parsing module is divided into two parts: validation of the URL and subsequent parsing of said URL. The implementation of this module can be found in the parseUrl.h and parseUrl.c files.
The URL validation is made using POSIX’s implementation of regular expressions (regex for short). Two regular expressions are used, one for authenticated login and one for anonymous login:

	\begin{lstlisting}[language=C, breaklines=true]
	char* regex_auth = "ftp://[A-Za-z0-9]+:[A-Za-z0-9]+@[A-Za-z0-9._~:?#!$&'()*+,:;=-]+/[A-Za-z0-9._~:/?#@!$&'()*+,:;=-]+";
	char* regex_auth_anon = "ftp://[A-Za-z0-9._~:?#!$&'()*+,:;=-]+/[A-Za-z0-9._~:/?#@!$&'()*+,:;=-]+";
	\end{lstlisting}

	The URL validation is done in the following functions:

	\begin{lstlisting}[language=C, breaklines=true]
	int validateURL(char* url, int size);
	int validateURLAnon(char* url, int size);
	\end{lstlisting}

	The first function is public and the one called by main() to validate a URL. The second function is used only in the module, called by validateURL() in case the authenticated connection regex fails. Each one uses its respective regex. validateURL() returns 0 on authenticated connection, 1 on anonymous connection, 2 on invalid URL and -1 on an error situation.
	
	The parsing is done with the following function:

	\begin{lstlisting}[language=C, breaklines=true]
	void parseURL(char* url, int size, char* host, char* user, char* password, char* path, int anon);
	\end{lstlisting}

	The anon variable is a boolean variable which is equal to 1 if it’s a anonymous url or 0 otherwise. The host, user, password and path are filled with their respective values, being the password asked to the user in case of an anonymous login.

	\subsubsection{The FTP Client Module}

	The FTP Client Module can be found in the ftpClient.c and ftpClient.h files. After parsing the URL, the information is sent to the FTP Client module, which then does the following steps:

	\begin{enumerate}
		\item Connects to the FTP server through a TCP socket created in the ftp\textunderscore connect\textunderscore socket() function using the ftp\textunderscore connect() function.
		\item Logins to the FTP server using the ftp\textunderscore login\textunderscore host() function.
		\item Enters Passive mode and opens another TCP socket (the data socket) in the ftp\textunderscore set\textunderscore passive\textunderscore mode() function.
		\item Requests the file in the ftp\textunderscore retr\textunderscore file() function, downloading it in the ftp\textunderscore download\textunderscore file() function.
		\item Disconnects from the server and closes the TCP sockets with the ftp\textunderscore disconnect() function.
	\end{enumerate}

	Besides the stated functions, ftp\textunderscore send\textunderscore command() and ftp\textunderscore read\textunderscore answer() are used to send a command to the ftp server and read its reply, respectively. In order to connect to the server through the TCP socket, an IP address is required. So, the getIpAddress() function is used for said purpose.

	The FTP Client module contains a structure with the information necessary for every step described previously:

	\begin{lstlisting}[language=C, breaklines=true]
	typedef struct {
		char server_address[IP_MAX_SIZE]; //NNN.NNN.NNN.NNN
		int socketfd;
		int datafd;
		char username[MAX_STRING_SIZE];
		char password[MAX_STRING_SIZE];
		char path_to_file[MAX_STRING_SIZE];
		int file_size;
	} ftp_t;
	\end{lstlisting}

	The server\textunderscore address field contains the IP address of the FTP server provided by the user, translated by the function getIpAddress().
	
	The socketfd and datafd contain the file descriptors of the TCP sockets to communicate with the server. The username, password and path fields are the same from the URL Parsing module. The file\textunderscore size stores the file size given by the FTP server, used for calculating the current progress and to check the file’s integrity.

	\subsection{Report of a successful download}

We decided to perform two tests: an authenticated connection to the FTP server tom.fe.up.pt, requesting the file public\textunderscore html/reader/index.html and an anonymous connection to the FTP server speedtest.tele2.net requesting the file 1KB.zip.

The first test went as follows:

And the second test went as follows:


\begin{appendices}

%%%%%%%%%%%%%%%%%%%%%%%%%%%
% APENDICE - CODIGO FONTE %
%%%%%%%%%%%%%%%%%%%%%%%%%%%
\section{Source code}

\begin{Large}
linklayer.c:
\end{Large}

\vspace{15mm}

\end{appendices}

\end{document}
